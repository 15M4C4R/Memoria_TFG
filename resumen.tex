\chapter*{Resumen}
Durante la primera década del siglo XXI, los videojuegos de navegador representaron la principal puerta de entrada al ocio digital para una generación de jóvenes, democratizando el entretenimiento gracias a su inmediatez y accesibilidad. Plataformas impulsadas por tecnologías como \textit{Adobe Flash} permitieron que millones de usuarios disfrutaran de experiencias multijugador sencillas y directas, sin barreras de entrada ni hardware costoso. Sin embargo, los habitos de consumo fueron cambiando progresivamente debido al desplazamiento de la industria hacia grandes producciones y ecosistemas móviles cerrados, culminando con el cese definitivo del soporte de Flash Player en 2020 \cite{adobe_flash_eol}. El presente Trabajo Fin de Grado, \textit{King and Peasant}, nace de la motivación personal de recuperar esa esencia nostálgica, reconstruyendo la experiencia del juego social de navegador bajo los estándares de la ingeniería de software moderna. El objetivo no es competir con la industria comercial, sino crear un tributo tecnológico que devuelva al usuario la posibilidad de "entrar y jugar" con amigos de forma instantánea.

El desarrollo de la memoria comienza estableciendo el contexto histórico y la evolución tecnológica que justifica la pertinencia de revivir este formato en la web actual. A continuación, se detalla la metodología de gestión del proyecto y el análisis de requisitos, priorizando la experiencia de usuario y la interacción social sobre mecánicas complejas. Posteriormente, se profundiza en la implementación técnica del sistema, basado en el stack MERN (MariaDB, Express, React, Node.js) y WebSockets, poniendo especial énfasis en el apartado social y la sincronización de estados en tiempo real. Finalmente, se presentan las pruebas de validación y las conclusiones, demostrando cómo es posible evocar la nostalgia de los juegos clásicos utilizando una arquitectura robusta, escalable y segura.
