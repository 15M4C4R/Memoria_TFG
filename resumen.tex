\chapter*{Resumen}
Para una generación entera, la puerta de entrada al entretenimiento digital no requería potentes consolas ni largas instalaciones, sino simplemente una ventana del navegador. La infancia del autor estuvo definida por la inmediatez y accesibilidad de los juegos web, donde la diversión, ya fuera en solitario o compartida en línea, estaba a un solo clic de distancia. Sin embargo, con la obsolescencia de tecnologías como Flash, esa simplicidad se ha diluido en favor de ecosistemas más cerrados y complejos.

Este Trabajo Fin de Grado, titulado King and Peasant, nace de la motivación personal de recuperar esa esencia nostálgica, reconstruyendo la experiencia del juego de navegador clásico bajo los estándares de la ingeniería de software moderna. El proyecto presenta una plataforma multijugador de cartas en tiempo real desarrollada con una arquitectura MERN (MariaDB, Express, React, Node.js) y WebSockets. Más allá de la mecánica del juego, el sistema pone el foco en la interacción social que caracterizaba a aquellas comunidades, implementando búsqueda de usuarios, listas de amigos y salas de espera (lobbies) dinámicas. Todo ello, desplegado mediante contenedores Docker, demuestra cómo las tecnologías web actuales permiten revivir la magia del pasado con la seguridad, escalabilidad y rendimiento del presente.

