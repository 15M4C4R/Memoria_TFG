\chapter{Planificación}

\begin{quotation}[Novelist]{Ernest Hemingway (1899--1961)}
The good parts of a book may be only something a writer is lucky enough to overhear or it may be the wreck of his whole damn life -- and one is as good as the other.
\end{quotation}

\begin{abstract}
Resumen de lo que va a ocurrir en el capítulo. ¿Cuál es el objetivo que tenemos con este capítulo?
\end{abstract}

\section{Resumen temporal del proyecto}

\begin{table*}[htb]
    \centering
    \begin{coolTable}{ll}{2}
{Resumen del proyecto}
    \textbf{Fecha de inicio}&02/02/2026\\ % Basado en el nodo "2 - 8" de Febrero del canvas
    \textbf{Fecha de fin}&31/05/2026\\ % Basado en el nodo Mayo del canvas
    \textbf{Periodicidad de las revisiones}&Semanal (Sprints)\\
    \textbf{Carga de trabajo semanal}&XX horas\\ % Dato no disponible en el canvas
    \textbf{Horas totales previstas}&XX horas\\ % Dato no disponible en el canvas
    \textbf{Horas finales}&XX horas\\ % Dato no disponible en el canvas
    \end{coolTable}
    \caption{Tabla resumen de tiempos y planificación}
\end{table*}

\section{Planificación inicial}

Aquí un desglose de las iteraciones, comienzo y fin de cada una, basado en el calendario de desarrollo definido:

\begin{table*}[htb]
    \centering
    \begin{coolTable}{ll}{2}
{Resumen de iteraciones}
    \textbf{Iteración Febrero}&02/02/26 a 28/02/26\\
    \multicolumn{2}{l}{\footnotesize \textit{Tareas: Inicializar repositorio, Set Up Docker, Frontend/Backend, UML BD, Pantallas (Login, Registro, Inicio, Usuario).}} \\
    \textbf{Iteración Marzo}&01/03/26 a 31/03/26\\
    \multicolumn{2}{l}{\footnotesize \textit{Tareas: Pantalla de Sala y Listado, Lógica de Juego (Barajas, Cartas), Flujo de estados y turnos.}} \\
    \textbf{Iteración Abril}&01/04/26 a 30/04/26\\
    \multicolumn{2}{l}{\footnotesize \textit{Tareas: Sistema de Amistad, Chat en Partida, Continuación lógica de juego.}} \\
    \textbf{Iteración Mayo}&01/05/26 a 31/05/26\\
    \multicolumn{2}{l}{\footnotesize \textit{Tareas: Pantalla de Ganador, Refactorización, Pruebas finales, Finalizar memoria.}} \\
    \end{coolTable}
    \caption{Planificación temporal de iteraciones}
\end{table*}

Explicar cómo se han decidido las fechas, interacción con fechas importantes y situaciones personales.

\textbf{ESTE CAPÍTULO DEBE ESCRIBIRSE AL COMIENZO DEL PROYECTO}

\section{Informe de tiempos del proyecto}

Lo mismo que el anterior pero con datos reales. Ver Tabla \ref{tab:InformeTiempos}.

\begin{table*}[htb]
    \centering
    \begin{coolTable}{ll}{2}
{Resumen de iteraciones (Real)}
    \textbf{Iteración Febrero}&dd/mm/aa a dd/mm/aa\\
    \textbf{Iteración Marzo}&dd/mm/aa a dd/mm/aa\\
    \textbf{Iteración Abril}&dd/mm/aa a dd/mm/aa\\
    \textbf{Iteración Mayo}&dd/mm/aa a dd/mm/aa\\
    \end{coolTable}
    \caption{Ejecución real del proyecto\label{tab:InformeTiempos}}
\end{table*}

Justificar los retrasos de forma detallada aquí para cada una de las iteraciones. Explicar las razones.